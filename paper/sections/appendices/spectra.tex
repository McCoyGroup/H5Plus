%%%%%%%%%%%%%%%%%%%%%%%%%%%%%%%%%%%%%%%%%%%%%%%%%%%%%%%%%
%%
%%					PAPER CONFIG
%%
%%					This configures the paper environment.
%%					The biggest thing is to set the root directory appropriately.
%%%%%%%%%%%%%%%%%%%%%%%%%%%%%%%%%%%%%%%%%%%%%%%%%%%%%%%
%%
\newcommand*{\RootDirectory}{"/Users/Mark/Documents/UW/Research/H5+/paper"}

%%%%%%%%%%%%%%%%%%%%%%%%%%%%%%%%%%%%%%%%%%%%%%%%%%%%%%%%%%%%%%%%%%%%%
%% This is a (brief) model paper using the achemso class
%% The document class accepts keyval options, which should include
%% the target journal and optionally the manuscript type.
%%%%%%%%%%%%%%%%%%%%%%%%%%%%%%%%%%%%%%%%%%%%%%%%%%%%%%%%%%%%%%%%%%%%%
\documentclass[journal=jacsat, manuscript=article, layout=twocolumn]{achemso}

%%%%%%%%%%%%%%%%%%%%%%%%%%%%%%%%%%%%%%%%%%%%%%%%%%%%%%%%%%%%%%%%%%%%%
%% If issues arise when submitting your manuscript, you may want to
%% un-comment the next line.  This provides information on the
%% version of every file you have used.
%%%%%%%%%%%%%%%%%%%%%%%%%%%%%%%%%%%%%%%%%%%%%%%%%%%%%%%%%%%%%%%%%%%%%
% \listfiles

\usepackage{import}

\import{\RootDirectory}{config/packages.tex}

\import{\RootDirectory}{config/custom.tex}

\import{\RootDirectory}{config/author.tex}

\import{\RootDirectory}{config/title.tex}

\import{\RootDirectory}{config/keywords.tex}

%\begin{document}

The use of adiabatic surfaces adds a layer of complication onto our basic intensity equations. Our wavefunctions in the adiabatic approximation are given by Equation \refeq{adiabatic_wfns}. This mean that our oscillator strength will be given by \refeq{couple_spec_osc}.

\loadeq{couple_spec_osc}

Since our $\phi^{\mhtwo}$ and $\psi^{\mhplus}$ operate over different coordinates we can separate each of these integrals as shown in Equation \refeq{couple_spec_osc_sep}.

\loadeq{couple_spec_osc_sep}

We can then replace all the terms in $r_1$ and $r_2$ with an effective dipole moment in $a$ and $s$ for each surface as given in \refeq{spec_tmoment}. A sample of these surfaces is plotted in \reffig{dipole_surfaces}. This reduces \refeq{couple_spec_osc} to an expansion in terms looking like \refeq{spec_osc}.

\loadeq{spec_tmoment}

%\end{document}
