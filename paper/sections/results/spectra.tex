%%%%%%%%%%%%%%%%%%%%%%%%%%%%%%%%%%%%%%%%%%%%%%%%%%%%%%%%%
%%
%%					PAPER CONFIG
%%
%%					This configures the paper environment.
%%					The biggest thing is to set the root directory appropriately.
%%%%%%%%%%%%%%%%%%%%%%%%%%%%%%%%%%%%%%%%%%%%%%%%%%%%%%%
%%
\newcommand*{\RootDirectory}{"/Users/Mark/Documents/UW/Research/H5+/paper"}

%%%%%%%%%%%%%%%%%%%%%%%%%%%%%%%%%%%%%%%%%%%%%%%%%%%%%%%%%%%%%%%%%%%%%
%% This is a (brief) model paper using the achemso class
%% The document class accepts keyval options, which should include
%% the target journal and optionally the manuscript type.
%%%%%%%%%%%%%%%%%%%%%%%%%%%%%%%%%%%%%%%%%%%%%%%%%%%%%%%%%%%%%%%%%%%%%
\documentclass[journal=jacsat, manuscript=article, layout=twocolumn]{achemso}

%%%%%%%%%%%%%%%%%%%%%%%%%%%%%%%%%%%%%%%%%%%%%%%%%%%%%%%%%%%%%%%%%%%%%
%% If issues arise when submitting your manuscript, you may want to
%% un-comment the next line.  This provides information on the
%% version of every file you have used.
%%%%%%%%%%%%%%%%%%%%%%%%%%%%%%%%%%%%%%%%%%%%%%%%%%%%%%%%%%%%%%%%%%%%%
% \listfiles

\usepackage{import}

\import{\RootDirectory}{config/packages.tex}

\import{\RootDirectory}{config/custom.tex}

\import{\RootDirectory}{config/author.tex}

\import{\RootDirectory}{config/title.tex}

\import{\RootDirectory}{config/keywords.tex}

%\begin{document}

\TODO{Language here is really meh... try to make stuff more rigorous and well thought out}

From the wavefunctions we obtain we can also calculate a frequencies and intensities as described previously. The spectrum we obtain is shown with comparison to experiment in Figure \ref{fig:spec}. We can see that we do a reasonable job of reproducing both energies an intensities. In particular, as shown in \reffig{spec_chunk_1}, relative to prior work~\cite{Lin2012} we do a better job of reproducing the true frequencies of the spectrum while maintaining the previously calculated intensity pattern. We have also extended beyond this prior work to regions of the spectrum above \squigg$2600$ \wavenumbers{}. In the region of the spectrum corresponding to a single quantum in the outer \htwo{} stretches with small numbers of quanta in \hplus{} vibrations (\squigg$3500-4000$ \wavenumbers{}) we see we do a good job of reproducing both the frequencies and intensities of the major transitions. This is show in more detail in \reffig{spec_chunk_2}. Moreover, moving to an unpublished portion of the spectrum in the \squigg$4750-7500$ \wavenumbers{} range we place a significant amount of intensity in the regions of the spectrum that experimentally have the most intensity, as can be seen more clearly in \reffig{spec_chunk_3}. We have also assigned the states that are contributing here.

\loadfig{spec}

Before discussing these assignments, however, it is worth comparing to what we might expect to get without coupling of our adiabatic states. This may be thought of as our first-order approximation to the spectrum. This is shown in \reffig{no_coupling}. We can see that our frequencies are generally shifted further away from the peak positions than in the coupled representation and there are many more peaks that are missing intensity. Moving higher in the spectrum to the range shown in \reffig{no_coupling_higher} we see a similar thing occurring in the \squigg$6800-7500$ \wavenumbers{} range which suggests that we will want to also perform the same coupling procedure for the next triplet of adiabatic states as well.

%\end{document}
