%%%%%%%%%%%%%%%%%%%%%%%%%%%%%%%%%%%%%%%%%%%%%%%%%%%%%%%%%
%%
%%					PAPER CONFIG
%%
%%					This configures the paper environment.
%%					The biggest thing is to set the root directory appropriately.
%%%%%%%%%%%%%%%%%%%%%%%%%%%%%%%%%%%%%%%%%%%%%%%%%%%%%%%
%%
\newcommand*{\RootDirectory}{"/Users/Mark/Documents/UW/Research/H5+/paper"}

%%%%%%%%%%%%%%%%%%%%%%%%%%%%%%%%%%%%%%%%%%%%%%%%%%%%%%%%%%%%%%%%%%%%%
%% This is a (brief) model paper using the achemso class
%% The document class accepts keyval options, which should include
%% the target journal and optionally the manuscript type.
%%%%%%%%%%%%%%%%%%%%%%%%%%%%%%%%%%%%%%%%%%%%%%%%%%%%%%%%%%%%%%%%%%%%%
\documentclass[journal=jacsat, manuscript=article, layout=twocolumn]{achemso}

%%%%%%%%%%%%%%%%%%%%%%%%%%%%%%%%%%%%%%%%%%%%%%%%%%%%%%%%%%%%%%%%%%%%%
%% If issues arise when submitting your manuscript, you may want to
%% un-comment the next line.  This provides information on the
%% version of every file you have used.
%%%%%%%%%%%%%%%%%%%%%%%%%%%%%%%%%%%%%%%%%%%%%%%%%%%%%%%%%%%%%%%%%%%%%
% \listfiles

\usepackage{import}

\import{\RootDirectory}{config/packages.tex}

\import{\RootDirectory}{config/custom.tex}

\import{\RootDirectory}{config/author.tex}

\import{\RootDirectory}{config/title.tex}

\import{\RootDirectory}{config/keywords.tex}

%\begin{document}

From our wavefunctions we have also determined which \hplus{} and \htwos{} states contribute to the most intense peaks in the spectrum. Previous work has found the \hfive{} supports an abnormally high progression of excitations in the \hplus{} motion with significant transition intensity. Our findings further support this. The dominant \hplus{} state for each projection of our wavefunctions along with its intensity has been tabulated in \reftab{state_contribs}. We find that states with significant intensity progress as far out as \squigg$9$ quanta of excitation in the \hplus{} mode of each projection. This is in line with the \squigg$9$ quanta of excitation found in the \hplus{} mode with no quanta in \htwo{} modes by Lin et al.~\cite{Lin2012} and has similar physical implications for the $\mhtwo + \mhthree \rightleftharpoons \mhfive$ reaction\COM{should I discuss these...Anne and Zhou discussed them at reasonable length in their 2012 paper...?}, although now over a base excitation in the \htwo{}s rather than being purely in the \hplus{} motion.
%\end{document}
