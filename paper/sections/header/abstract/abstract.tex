
The \hfive{} molecule is a an atronomically relevant molecule that readily samples large numbers of minima on its potential energy landscape. This would appear to provide serious difficulties with respect to modeling \hfive{}, but we show we can use a reduced dimensional approach and still obtain a reasonably accurate spectrum. We study the vibrations of \hfive{} via a Hamiltonian that only models four of the nine vibrations in \hfive{} and are still able to reproduce the intensity pattern of the spectrum up to \squigg$5500$ \wavenumbers{}.  The states that carry significant intensity in the higher frequency portion of the spectrum show \hplus{} modes with up to $9-10$ quanta of excitation. Such high excitations have been shown by theory for the ground state of \hfive{} and we show that the same happens even above the dissociation energy of \hfive{} along the $\mhtwo + \mhthree \rightleftharpoons \mhfive$ channel.
