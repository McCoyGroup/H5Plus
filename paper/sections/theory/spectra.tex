% %%%%%%%%%%%%%%%%%%%%%%%%%%%%%%%%%%%%%%%%%%%%%%%%%%%%%%%%
%%
%%					PAPER CONFIG
%%
%%					This configures the paper environment.
%%					The biggest thing is to set the root directory appropriately.
%%%%%%%%%%%%%%%%%%%%%%%%%%%%%%%%%%%%%%%%%%%%%%%%%%%%%%%
%%
\newcommand*{\RootDirectory}{"/Users/Mark/Documents/UW/Research/H5+/paper"}

%%%%%%%%%%%%%%%%%%%%%%%%%%%%%%%%%%%%%%%%%%%%%%%%%%%%%%%%%%%%%%%%%%%%%
%% This is a (brief) model paper using the achemso class
%% The document class accepts keyval options, which should include
%% the target journal and optionally the manuscript type.
%%%%%%%%%%%%%%%%%%%%%%%%%%%%%%%%%%%%%%%%%%%%%%%%%%%%%%%%%%%%%%%%%%%%%
\documentclass[journal=jacsat, manuscript=article, layout=twocolumn]{achemso}

%%%%%%%%%%%%%%%%%%%%%%%%%%%%%%%%%%%%%%%%%%%%%%%%%%%%%%%%%%%%%%%%%%%%%
%% If issues arise when submitting your manuscript, you may want to
%% un-comment the next line.  This provides information on the
%% version of every file you have used.
%%%%%%%%%%%%%%%%%%%%%%%%%%%%%%%%%%%%%%%%%%%%%%%%%%%%%%%%%%%%%%%%%%%%%
% \listfiles

\usepackage{import}

\import{\RootDirectory}{config/packages.tex}

\import{\RootDirectory}{config/custom.tex}

\import{\RootDirectory}{config/author.tex}

\import{\RootDirectory}{config/title.tex}

\import{\RootDirectory}{config/keywords.tex}

% \begin{document}

In general, the intensity of a transition is proportional the oscillator strength of the transition and the frequency associated with the transition. The basic form of this is given in \refeq{spec_eqs}.

\loadeq{spec_eqs}

The oscillator strength is given in \refeq{spec_osc}.
\loadeq{spec_osc}

For our particular case, since we solved our problem under an adiabatic approximation we introduce a mild complication to our equations. The details of this are worked out in \refapp{spectra} but analogously to the effective potential energy surfaces for the \hplus{} motion we generate effective dipole surfaces for the \hplus{} transitions. A sample of these surfaces are plotted in \reffig{dipole_surfaces}.

\loadfig{dipole_surfaces}

The symmetry of these functions reflects the symmetry of adiabatic state. The ground adiabatic state is symmetric about $a$ and thus the effective dipole surface is asymmetric about $a$. The same holds for the third adiabatic state while the opposite is the case for the second.

% \end{document}
