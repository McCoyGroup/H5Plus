%%%%%%%%%%%%%%%%%%%%%%%%%%%%%%%%%%%%%%%%%%%%%%%%%%%%%%%%%
%%
%%					PAPER CONFIG
%%
%%					This configures the paper environment.
%%					The biggest thing is to set the root directory appropriately.
%%%%%%%%%%%%%%%%%%%%%%%%%%%%%%%%%%%%%%%%%%%%%%%%%%%%%%%
%%
\newcommand*{\RootDirectory}{"/Users/Mark/Documents/UW/Research/H5+/paper"}

%%%%%%%%%%%%%%%%%%%%%%%%%%%%%%%%%%%%%%%%%%%%%%%%%%%%%%%%%%%%%%%%%%%%%
%% This is a (brief) model paper using the achemso class
%% The document class accepts keyval options, which should include
%% the target journal and optionally the manuscript type.
%%%%%%%%%%%%%%%%%%%%%%%%%%%%%%%%%%%%%%%%%%%%%%%%%%%%%%%%%%%%%%%%%%%%%
\documentclass[journal=jacsat, manuscript=article, layout=twocolumn]{achemso}

%%%%%%%%%%%%%%%%%%%%%%%%%%%%%%%%%%%%%%%%%%%%%%%%%%%%%%%%%%%%%%%%%%%%%
%% If issues arise when submitting your manuscript, you may want to
%% un-comment the next line.  This provides information on the
%% version of every file you have used.
%%%%%%%%%%%%%%%%%%%%%%%%%%%%%%%%%%%%%%%%%%%%%%%%%%%%%%%%%%%%%%%%%%%%%
% \listfiles

\usepackage{import}

\import{\RootDirectory}{config/packages.tex}

\import{\RootDirectory}{config/custom.tex}

\import{\RootDirectory}{config/author.tex}

\import{\RootDirectory}{config/title.tex}

\import{\RootDirectory}{config/keywords.tex}

%\begin{document}

We treat the \hplus{} and \htwo{} motions adiabatically. The \htwo{} fundamental frequency is on the order of ~3500 \wavenumbers{} while the \hplus{} fundamental is on the order of ~500 \wavenumbers{}, thus\COM{ by analogy to the Born-Oppenheimer approximation} we may assume a significant separation of timescales of the two motions. We may then use this assumption to reduce our four dimensional problem to a pair of coupled two dimensional problems as in \refeq{adiabatic_eqs}.

\loadeq{adiabatic_eqs}

Conceptually, we are creating a set of effective potential energy surfaces for the \hplus{} by solving for the energies of the \htwo{}s at every \hplus{} position. Some of these surfaces may be seen in \reffig{adiabatic_surfaces}.

\loadfig{adiabatic_surfaces}

This separation of timescales allows us to very cleanly express our wavefunctions as contribution from an \hplus{} vibration and a \htwo{} vibration and provides the added benefit of reducing the size of the matrices we need to work with. Many techniques, ours included, scale poorly with dimension and so a pair of coupled 2D problems can be solved much more quickly than a single 4D problem. However, this approximation can break down if the adiabatic surfaces we generate don't truly have the required separation. In particular, for the outer \htwo{} states the wavefunctions come as symmetric and anti-symmetric combinations of states exciting each of the \htwo{}s individually. These combinations are shown in \refeq{htwo_states}.

\loadeq{htwo_states}

The symmetric and anti-symmetric combinations of these two states are very close in energy as can be seen in \reffig{adiabat_cuts}. Thus we should expect that the \hplus{} state will really experience a potential that is combination of the surfaces generated by these two states. The details of this coupling are discussed in \refapp{adiabatic_coupling}.

\loadfig{adiabat_cuts}

The resultant forms of the wavefunctions are given in \refeq{adiabatic_wfns}.

\loadeq{adiabatic_wfns}

%\end{document}
