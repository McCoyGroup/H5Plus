% %%%%%%%%%%%%%%%%%%%%%%%%%%%%%%%%%%%%%%%%%%%%%%%%%%%%%%%%
%%
%%					PAPER CONFIG
%%
%%					This configures the paper environment.
%%					The biggest thing is to set the root directory appropriately.
%%%%%%%%%%%%%%%%%%%%%%%%%%%%%%%%%%%%%%%%%%%%%%%%%%%%%%%
%%
\newcommand*{\RootDirectory}{"/Users/Mark/Documents/UW/Research/H5+/paper"}

%%%%%%%%%%%%%%%%%%%%%%%%%%%%%%%%%%%%%%%%%%%%%%%%%%%%%%%%%%%%%%%%%%%%%
%% This is a (brief) model paper using the achemso class
%% The document class accepts keyval options, which should include
%% the target journal and optionally the manuscript type.
%%%%%%%%%%%%%%%%%%%%%%%%%%%%%%%%%%%%%%%%%%%%%%%%%%%%%%%%%%%%%%%%%%%%%
\documentclass[journal=jacsat, manuscript=article, layout=twocolumn]{achemso}

%%%%%%%%%%%%%%%%%%%%%%%%%%%%%%%%%%%%%%%%%%%%%%%%%%%%%%%%%%%%%%%%%%%%%
%% If issues arise when submitting your manuscript, you may want to
%% un-comment the next line.  This provides information on the
%% version of every file you have used.
%%%%%%%%%%%%%%%%%%%%%%%%%%%%%%%%%%%%%%%%%%%%%%%%%%%%%%%%%%%%%%%%%%%%%
% \listfiles

\usepackage{import}

\import{\RootDirectory}{config/packages.tex}

\import{\RootDirectory}{config/custom.tex}

\import{\RootDirectory}{config/author.tex}

\import{\RootDirectory}{config/title.tex}

\import{\RootDirectory}{config/keywords.tex}

% \begin{document}

\COM{Because we see experimental and theoretical signatures of predominantly shared \hplus{} and outer \htwo{} motions in the spectra}
Taking these four coordinates, we construct our Hamiltonian in terms of the stretches of the \htwo{}s and of the shared \hplus{}. We focus on the $C_{2v}$ equilibrium geometry and $D_{2d}$ vibrationally averaged geometry in our calculation. The $D_{2d}$ structure and overlaid coordinate system is shown in \reffig{h5geom}. The Hamiltonian we are using in these coordinates is given in \refeq{baseHam}.

\COM{This might actually not be right...}
\loadeq{baseHam}

We then symmetrize our $R_1$ and $R_2$ coordinates by the transformations in \refeq{symm}.

\loadeq{symm}

This then decouples our shared proton coordinates, giving us our working Hamiltonian, which is \refeq{hamm}.

\loadeq{hamm}

In this $\mu_{a}$ and $\mu_{s}$ are the effective reduced masses of the symmetrized stretches and $\mu_{r_1}$ and $\mu_{r_2}$ are those of a basic \htwo{} stretch. The expressions for these in terms of the mass of hydrogen $m_{\mathrm{H}}$ are given in \refeq{masses}.

\loadeq{masses}

The potential $V(a, s, r_1, r_2)$ is constructed by using the Gaussian 09 software package and performing electronic structure calculations on a regular grid in the four coordinates at the aug-cc-PVDZ level of theory. 
% \end{document}
