%%%%%%%%%%%%%%%%%%%%%%%%%%%%%%%%%%%%%%%%%%%%%%%%%%%%%%%%%
%%
%%					PAPER CONFIG
%%
%%					This configures the paper environment.
%%					The biggest thing is to set the root directory appropriately.
%%%%%%%%%%%%%%%%%%%%%%%%%%%%%%%%%%%%%%%%%%%%%%%%%%%%%%%
%%
\newcommand*{\RootDirectory}{"/Users/Mark/Documents/UW/Research/H5+/paper"}

%%%%%%%%%%%%%%%%%%%%%%%%%%%%%%%%%%%%%%%%%%%%%%%%%%%%%%%%%%%%%%%%%%%%%
%% This is a (brief) model paper using the achemso class
%% The document class accepts keyval options, which should include
%% the target journal and optionally the manuscript type.
%%%%%%%%%%%%%%%%%%%%%%%%%%%%%%%%%%%%%%%%%%%%%%%%%%%%%%%%%%%%%%%%%%%%%
\documentclass[journal=jacsat, manuscript=article, layout=twocolumn]{achemso}

%%%%%%%%%%%%%%%%%%%%%%%%%%%%%%%%%%%%%%%%%%%%%%%%%%%%%%%%%%%%%%%%%%%%%
%% If issues arise when submitting your manuscript, you may want to
%% un-comment the next line.  This provides information on the
%% version of every file you have used.
%%%%%%%%%%%%%%%%%%%%%%%%%%%%%%%%%%%%%%%%%%%%%%%%%%%%%%%%%%%%%%%%%%%%%
% \listfiles

\usepackage{import}

\import{\RootDirectory}{config/packages.tex}

\import{\RootDirectory}{config/custom.tex}

\import{\RootDirectory}{config/author.tex}

\import{\RootDirectory}{config/title.tex}

\import{\RootDirectory}{config/keywords.tex}

%\begin{document}

Moving forward, we still need to include the states with two quanta in the \htwo{} stretches. These will likely be in the \squigg$6500-7500$ \wavenumbers{} range and will module the adiabatic peaks, potentially with similar effect to what was seen over the \squigg$3500-5500$ \wavenumbers{} range.\COM{It may be the case that we need to couple the one quantum to the two quanta states as well, as from a pure intensity borrowing perspective we do not have enough intensity in that range to in the single state approximation to borrow from.} Alongside this we need to obtain state assignments for these higher-energy coupled states. These will determine to what degree the long progression in \hfive{} is a function of the \htwo{} as well as the \hplus{}.\TODO{what else was there to do with this project...?} We also note that the frequencies we have obtained differ from experiment by a non-neglible margin. Some portion of this difference comes from the fact that we are using a reduced dimensional Hamiltonian and can thus never expect to get the true ground state. On the other hand, by evaluating our wavefunctions over a larger grid and increasing our basis size we can expect to recover at least some portion of this difference. Given that our frequencies and intensities in the \squigg$3500-5500$ \wavenumbers{} range are already in general agreement with experiment, this should allow us to compare the degree to which the \hplus{} states on the higher adiabatic surfaces and those on the lowest adiabatic surface respond to a refinement in the calculation.\COM{This can give us insight into the tightness of our wavefunctions and thus into the distribution of \hplus{} positions in reality.}
%\end{document}
